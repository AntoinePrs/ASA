% Options for packages loaded elsewhere
% Options for packages loaded elsewhere
\PassOptionsToPackage{unicode}{hyperref}
\PassOptionsToPackage{hyphens}{url}
\PassOptionsToPackage{dvipsnames,svgnames,x11names}{xcolor}
%
\documentclass[
  french,
  letterpaper,
  DIV=11,
  numbers=noendperiod]{scrartcl}
\usepackage{xcolor}
\usepackage{amsmath,amssymb}
\setcounter{secnumdepth}{-\maxdimen} % remove section numbering
\usepackage{iftex}
\ifPDFTeX
  \usepackage[T1]{fontenc}
  \usepackage[utf8]{inputenc}
  \usepackage{textcomp} % provide euro and other symbols
\else % if luatex or xetex
  \usepackage{unicode-math} % this also loads fontspec
  \defaultfontfeatures{Scale=MatchLowercase}
  \defaultfontfeatures[\rmfamily]{Ligatures=TeX,Scale=1}
\fi
\usepackage{lmodern}
\ifPDFTeX\else
  % xetex/luatex font selection
\fi
% Use upquote if available, for straight quotes in verbatim environments
\IfFileExists{upquote.sty}{\usepackage{upquote}}{}
\IfFileExists{microtype.sty}{% use microtype if available
  \usepackage[]{microtype}
  \UseMicrotypeSet[protrusion]{basicmath} % disable protrusion for tt fonts
}{}
\makeatletter
\@ifundefined{KOMAClassName}{% if non-KOMA class
  \IfFileExists{parskip.sty}{%
    \usepackage{parskip}
  }{% else
    \setlength{\parindent}{0pt}
    \setlength{\parskip}{6pt plus 2pt minus 1pt}}
}{% if KOMA class
  \KOMAoptions{parskip=half}}
\makeatother
% Make \paragraph and \subparagraph free-standing
\makeatletter
\ifx\paragraph\undefined\else
  \let\oldparagraph\paragraph
  \renewcommand{\paragraph}{
    \@ifstar
      \xxxParagraphStar
      \xxxParagraphNoStar
  }
  \newcommand{\xxxParagraphStar}[1]{\oldparagraph*{#1}\mbox{}}
  \newcommand{\xxxParagraphNoStar}[1]{\oldparagraph{#1}\mbox{}}
\fi
\ifx\subparagraph\undefined\else
  \let\oldsubparagraph\subparagraph
  \renewcommand{\subparagraph}{
    \@ifstar
      \xxxSubParagraphStar
      \xxxSubParagraphNoStar
  }
  \newcommand{\xxxSubParagraphStar}[1]{\oldsubparagraph*{#1}\mbox{}}
  \newcommand{\xxxSubParagraphNoStar}[1]{\oldsubparagraph{#1}\mbox{}}
\fi
\makeatother

\usepackage{color}
\usepackage{fancyvrb}
\newcommand{\VerbBar}{|}
\newcommand{\VERB}{\Verb[commandchars=\\\{\}]}
\DefineVerbatimEnvironment{Highlighting}{Verbatim}{commandchars=\\\{\}}
% Add ',fontsize=\small' for more characters per line
\usepackage{framed}
\definecolor{shadecolor}{RGB}{241,243,245}
\newenvironment{Shaded}{\begin{snugshade}}{\end{snugshade}}
\newcommand{\AlertTok}[1]{\textcolor[rgb]{0.68,0.00,0.00}{#1}}
\newcommand{\AnnotationTok}[1]{\textcolor[rgb]{0.37,0.37,0.37}{#1}}
\newcommand{\AttributeTok}[1]{\textcolor[rgb]{0.40,0.45,0.13}{#1}}
\newcommand{\BaseNTok}[1]{\textcolor[rgb]{0.68,0.00,0.00}{#1}}
\newcommand{\BuiltInTok}[1]{\textcolor[rgb]{0.00,0.23,0.31}{#1}}
\newcommand{\CharTok}[1]{\textcolor[rgb]{0.13,0.47,0.30}{#1}}
\newcommand{\CommentTok}[1]{\textcolor[rgb]{0.37,0.37,0.37}{#1}}
\newcommand{\CommentVarTok}[1]{\textcolor[rgb]{0.37,0.37,0.37}{\textit{#1}}}
\newcommand{\ConstantTok}[1]{\textcolor[rgb]{0.56,0.35,0.01}{#1}}
\newcommand{\ControlFlowTok}[1]{\textcolor[rgb]{0.00,0.23,0.31}{\textbf{#1}}}
\newcommand{\DataTypeTok}[1]{\textcolor[rgb]{0.68,0.00,0.00}{#1}}
\newcommand{\DecValTok}[1]{\textcolor[rgb]{0.68,0.00,0.00}{#1}}
\newcommand{\DocumentationTok}[1]{\textcolor[rgb]{0.37,0.37,0.37}{\textit{#1}}}
\newcommand{\ErrorTok}[1]{\textcolor[rgb]{0.68,0.00,0.00}{#1}}
\newcommand{\ExtensionTok}[1]{\textcolor[rgb]{0.00,0.23,0.31}{#1}}
\newcommand{\FloatTok}[1]{\textcolor[rgb]{0.68,0.00,0.00}{#1}}
\newcommand{\FunctionTok}[1]{\textcolor[rgb]{0.28,0.35,0.67}{#1}}
\newcommand{\ImportTok}[1]{\textcolor[rgb]{0.00,0.46,0.62}{#1}}
\newcommand{\InformationTok}[1]{\textcolor[rgb]{0.37,0.37,0.37}{#1}}
\newcommand{\KeywordTok}[1]{\textcolor[rgb]{0.00,0.23,0.31}{\textbf{#1}}}
\newcommand{\NormalTok}[1]{\textcolor[rgb]{0.00,0.23,0.31}{#1}}
\newcommand{\OperatorTok}[1]{\textcolor[rgb]{0.37,0.37,0.37}{#1}}
\newcommand{\OtherTok}[1]{\textcolor[rgb]{0.00,0.23,0.31}{#1}}
\newcommand{\PreprocessorTok}[1]{\textcolor[rgb]{0.68,0.00,0.00}{#1}}
\newcommand{\RegionMarkerTok}[1]{\textcolor[rgb]{0.00,0.23,0.31}{#1}}
\newcommand{\SpecialCharTok}[1]{\textcolor[rgb]{0.37,0.37,0.37}{#1}}
\newcommand{\SpecialStringTok}[1]{\textcolor[rgb]{0.13,0.47,0.30}{#1}}
\newcommand{\StringTok}[1]{\textcolor[rgb]{0.13,0.47,0.30}{#1}}
\newcommand{\VariableTok}[1]{\textcolor[rgb]{0.07,0.07,0.07}{#1}}
\newcommand{\VerbatimStringTok}[1]{\textcolor[rgb]{0.13,0.47,0.30}{#1}}
\newcommand{\WarningTok}[1]{\textcolor[rgb]{0.37,0.37,0.37}{\textit{#1}}}

\usepackage{longtable,booktabs,array}
\usepackage{calc} % for calculating minipage widths
% Correct order of tables after \paragraph or \subparagraph
\usepackage{etoolbox}
\makeatletter
\patchcmd\longtable{\par}{\if@noskipsec\mbox{}\fi\par}{}{}
\makeatother
% Allow footnotes in longtable head/foot
\IfFileExists{footnotehyper.sty}{\usepackage{footnotehyper}}{\usepackage{footnote}}
\makesavenoteenv{longtable}
\usepackage{graphicx}
\makeatletter
\newsavebox\pandoc@box
\newcommand*\pandocbounded[1]{% scales image to fit in text height/width
  \sbox\pandoc@box{#1}%
  \Gscale@div\@tempa{\textheight}{\dimexpr\ht\pandoc@box+\dp\pandoc@box\relax}%
  \Gscale@div\@tempb{\linewidth}{\wd\pandoc@box}%
  \ifdim\@tempb\p@<\@tempa\p@\let\@tempa\@tempb\fi% select the smaller of both
  \ifdim\@tempa\p@<\p@\scalebox{\@tempa}{\usebox\pandoc@box}%
  \else\usebox{\pandoc@box}%
  \fi%
}
% Set default figure placement to htbp
\def\fps@figure{htbp}
\makeatother



\ifLuaTeX
\usepackage[bidi=basic]{babel}
\else
\usepackage[bidi=default]{babel}
\fi
% get rid of language-specific shorthands (see #6817):
\let\LanguageShortHands\languageshorthands
\def\languageshorthands#1{}


\setlength{\emergencystretch}{3em} % prevent overfull lines

\providecommand{\tightlist}{%
  \setlength{\itemsep}{0pt}\setlength{\parskip}{0pt}}



 


\KOMAoption{captions}{tableheading}
\makeatletter
\@ifpackageloaded{tcolorbox}{}{\usepackage[skins,breakable]{tcolorbox}}
\@ifpackageloaded{fontawesome5}{}{\usepackage{fontawesome5}}
\definecolor{quarto-callout-color}{HTML}{909090}
\definecolor{quarto-callout-note-color}{HTML}{0758E5}
\definecolor{quarto-callout-important-color}{HTML}{CC1914}
\definecolor{quarto-callout-warning-color}{HTML}{EB9113}
\definecolor{quarto-callout-tip-color}{HTML}{00A047}
\definecolor{quarto-callout-caution-color}{HTML}{FC5300}
\definecolor{quarto-callout-color-frame}{HTML}{acacac}
\definecolor{quarto-callout-note-color-frame}{HTML}{4582ec}
\definecolor{quarto-callout-important-color-frame}{HTML}{d9534f}
\definecolor{quarto-callout-warning-color-frame}{HTML}{f0ad4e}
\definecolor{quarto-callout-tip-color-frame}{HTML}{02b875}
\definecolor{quarto-callout-caution-color-frame}{HTML}{fd7e14}
\makeatother
\makeatletter
\@ifpackageloaded{caption}{}{\usepackage{caption}}
\AtBeginDocument{%
\ifdefined\contentsname
  \renewcommand*\contentsname{Table des matières}
\else
  \newcommand\contentsname{Table des matières}
\fi
\ifdefined\listfigurename
  \renewcommand*\listfigurename{Liste des Figures}
\else
  \newcommand\listfigurename{Liste des Figures}
\fi
\ifdefined\listtablename
  \renewcommand*\listtablename{Liste des Tables}
\else
  \newcommand\listtablename{Liste des Tables}
\fi
\ifdefined\figurename
  \renewcommand*\figurename{Figure}
\else
  \newcommand\figurename{Figure}
\fi
\ifdefined\tablename
  \renewcommand*\tablename{Table}
\else
  \newcommand\tablename{Table}
\fi
}
\@ifpackageloaded{float}{}{\usepackage{float}}
\floatstyle{ruled}
\@ifundefined{c@chapter}{\newfloat{codelisting}{h}{lop}}{\newfloat{codelisting}{h}{lop}[chapter]}
\floatname{codelisting}{Listing}
\newcommand*\listoflistings{\listof{codelisting}{Liste des Listings}}
\makeatother
\makeatletter
\makeatother
\makeatletter
\@ifpackageloaded{caption}{}{\usepackage{caption}}
\@ifpackageloaded{subcaption}{}{\usepackage{subcaption}}
\makeatother
\usepackage{bookmark}
\IfFileExists{xurl.sty}{\usepackage{xurl}}{} % add URL line breaks if available
\urlstyle{same}
\hypersetup{
  pdftitle={Fouille de données et création d'objets spatiaux dans R},
  pdfauthor={Antoine Peris},
  pdflang={fr},
  colorlinks=true,
  linkcolor={blue},
  filecolor={Maroon},
  citecolor={Blue},
  urlcolor={Blue},
  pdfcreator={LaTeX via pandoc}}


\title{Fouille de données et création d'objets spatiaux dans R}
\author{Antoine Peris}
\date{}
\begin{document}
\maketitle


\subsection{Objectif}\label{objectif}

Cette séance a pour objectif de se remémorer des éléments de syntaxe de
R (le chargement des données, la manipulation de tableaux, etc.) et de
découvrir deux familles d'outils très utiles dans la perspective
d'acquérir et de gérer des données spatiales :

\begin{itemize}
\item
  les expressions régulières (qui permettent d'aller récupérer des
  informations de données textuelles peu structurées)
\item
  les objets spatiaux vectoriels (qui permettent de réaliser des
  traitements géomatiques en R, pré-requis pour de l'analyse spatiale)
\end{itemize}

Pour cela, nous travaillerons sur un projet concret : la construction
d'un tableau de données sur les immeubles en péril à Bordeaux à partir
de arrêtés publiés par le conseil municipal de la ville.

Nous partons de données brutes : le fichier est téléchargeable
\href{https://datahub.bordeaux-metropole.fr/explore/dataset/bor_arretes/export/}{ici}.

\subsection{Chargement et inspection des
données}\label{chargement-et-inspection-des-donnuxe9es}

Pour le chargement d'un fichier \texttt{.csv}, nous allons utiliser le
package \textbf{\{readr\}} qui optimise la lecture des fichiers. On
charge simultanément \textbf{\{dplyr\}} qui facilite la manipulation de
données tabulaires.

\begin{Shaded}
\begin{Highlighting}[]
\FunctionTok{library}\NormalTok{(readr)}
\FunctionTok{library}\NormalTok{(dplyr)}
\end{Highlighting}
\end{Shaded}

Le fichier utilise le point-virgule comme séparateur (comme souvent pour
les fichiers issus de producteur de données français), il faut donc le
lire avec \texttt{read\_csv2()} plutôt qu'avec \texttt{read\_csv()}.

\begin{Shaded}
\begin{Highlighting}[]
\NormalTok{arretes }\OtherTok{\textless{}{-}} \FunctionTok{read\_csv2}\NormalTok{(}\StringTok{"../data/bor\_affichage{-}reglementaire.csv"}\NormalTok{)}
\end{Highlighting}
\end{Shaded}

On peut regarder d'un coup d'œil les dimensions et le contenu du
tableau.

\begin{Shaded}
\begin{Highlighting}[]
\FunctionTok{glimpse}\NormalTok{(arretes)}
\end{Highlighting}
\end{Shaded}

\begin{verbatim}
Rows: 43,395
Columns: 8
$ Collectivité                       <chr> "Ville de Bordeaux", "Ville de Bord~
$ `Nature de l'acte`                 <chr> "Arrêté", "Arrêté", "Arrêté", "Arrê~
$ `Numéro de l'acte`                 <chr> "202132825", "202133003", "20213300~
$ `Intitulé de l'acte`               <chr> "Arrêté temporaire de circulation p~
$ `Date d'affichage`                 <date> 2021-12-17, 2021-12-17, 2021-12-17~
$ Trigramme                          <chr> "BOR", "BOR", "BOR", "BOR", "BOR", ~
$ `Accès au document`                <chr> "https://cdn.scnbdx.fr/web-afresco/~
$ `Accès document avec DCP (2 mois)` <chr> NA, NA, NA, NA, NA, NA, NA, NA, NA,~
\end{verbatim}

On visualise ensuite les premières lignes.

\begin{Shaded}
\begin{Highlighting}[]
\FunctionTok{head}\NormalTok{(arretes) }\SpecialCharTok{\%\textgreater{}\%}\NormalTok{ knitr}\SpecialCharTok{::}\FunctionTok{kable}\NormalTok{()}
\end{Highlighting}
\end{Shaded}

\begin{longtable}[]{@{}
  >{\raggedright\arraybackslash}p{(\linewidth - 14\tabcolsep) * \real{0.0499}}
  >{\raggedright\arraybackslash}p{(\linewidth - 14\tabcolsep) * \real{0.0471}}
  >{\raggedright\arraybackslash}p{(\linewidth - 14\tabcolsep) * \real{0.0471}}
  >{\raggedright\arraybackslash}p{(\linewidth - 14\tabcolsep) * \real{0.5014}}
  >{\raggedright\arraybackslash}p{(\linewidth - 14\tabcolsep) * \real{0.0471}}
  >{\raggedright\arraybackslash}p{(\linewidth - 14\tabcolsep) * \real{0.0277}}
  >{\raggedright\arraybackslash}p{(\linewidth - 14\tabcolsep) * \real{0.1884}}
  >{\raggedright\arraybackslash}p{(\linewidth - 14\tabcolsep) * \real{0.0914}}@{}}
\toprule\noalign{}
\begin{minipage}[b]{\linewidth}\raggedright
Collectivité
\end{minipage} & \begin{minipage}[b]{\linewidth}\raggedright
Nature de l'acte
\end{minipage} & \begin{minipage}[b]{\linewidth}\raggedright
Numéro de l'acte
\end{minipage} & \begin{minipage}[b]{\linewidth}\raggedright
Intitulé de l'acte
\end{minipage} & \begin{minipage}[b]{\linewidth}\raggedright
Date d'affichage
\end{minipage} & \begin{minipage}[b]{\linewidth}\raggedright
Trigramme
\end{minipage} & \begin{minipage}[b]{\linewidth}\raggedright
Accès au document
\end{minipage} & \begin{minipage}[b]{\linewidth}\raggedright
Accès document avec DCP (2 mois)
\end{minipage} \\
\midrule\noalign{}
\endhead
\bottomrule\noalign{}
\endlastfoot
Ville de Bordeaux & Arrêté & 202132825 & Arrêté temporaire de
circulation pour travaux-Quartier n°3 - Bordeaux Centre-Rue Ausone, de
la Rue du Chai des Farines jusqu'au Cours d'Alsace Et Lorraine-1 -
Circulation interdite & 2021-12-17 & BOR &
https://cdn.scnbdx.fr/web-afresco/BOGABOR/BOR\_ARR\_202132825.pdf &
NA \\
Ville de Bordeaux & Arrêté & 202133003 & Arrêté temporaire de
circulation pour travaux-Quartier n°3 - Bordeaux Centre-Rue Colbert, de
la Rue Turenne jusqu'à la Rue de la Franchise-1 - Circulation interdite
& 2021-12-17 & BOR &
https://cdn.scnbdx.fr/web-afresco/BOGABOR/BOR\_ARR\_202133003.pdf &
NA \\
Ville de Bordeaux & Arrêté & 202133005 & Arrêté temporaire de
circulation pour travaux-Quartier n°3 - Bordeaux Centre-49 Rue du Palais
Gallien-1 - Circulation interdite & 2021-12-17 & BOR &
https://cdn.scnbdx.fr/web-afresco/BOGABOR/BOR\_ARR\_202133005.pdf &
NA \\
Ville de Bordeaux & Arrêté & 202133430 & Arrêté temporaire de
circulation pour travaux-Quartier n°6 - Bordeaux Sud-du n° 2 au n° 4 rue
Vilaris et du n° 13 place Dormoy au n° 11 rue Vilaris -1 - Mise en
impasse & 2021-12-20 & BOR &
https://cdn.scnbdx.fr/web-afresco/BOGABOR/BOR\_ARR\_202133430.pdf &
NA \\
Ville de Bordeaux & Arrêté & 22BORPT00042 & ARRÊTÉ TEMPORAIRE DE
CIRCULATION - Quartier n°2 - Chartrons/Grand Parc/Jardin Public-Rue
Saint Laurent, du 1 jusqu'à la Rue Albert de Mun-1 - Circulation
interdite & 2022-02-15 & BOR &
https://cdn.scnbdx.fr/web-afresco/ADRDSCO/BOR\_ADRD\_22BORPT00042.pdf &
NA \\
Ville de Bordeaux & Arrêté & 22BORPT00193 & ARRÊTÉ TEMPORAIRE DE
CIRCULATION - Quartier n°3 - Bordeaux Centre-36 Rue de la Benatte-1 -
Neutralisation de voie & 2022-02-16 & BOR &
https://cdn.scnbdx.fr/web-afresco/ADRDSCO/BOR\_ADRD\_22BORPT00193.pdf &
NA \\
\end{longtable}

On voit que l'essentiel des informations est contenu de manière
non-structurée dans la colonne
\textbf{\texttt{arretes\$\textasciigrave{}Intitulé\ de\ l’acte\textasciigrave{}}}
(qui possède au passage un nom peu pratique à manipuler dans du code).
Il y a par ailleurs des colonnes qui ne nous serviront pas.

On sélectionnera et renommera ainsi les colonnes :

\begin{Shaded}
\begin{Highlighting}[]
\NormalTok{arretes }\OtherTok{\textless{}{-}}\NormalTok{ arretes }\SpecialCharTok{\%\textgreater{}\%} 
  \FunctionTok{select}\NormalTok{(}\AttributeTok{nature=}\StringTok{\textasciigrave{}}\AttributeTok{Nature de l\textquotesingle{}acte}\StringTok{\textasciigrave{}}\NormalTok{, }
         \AttributeTok{numero=}\StringTok{\textasciigrave{}}\AttributeTok{Numéro de l\textquotesingle{}acte}\StringTok{\textasciigrave{}}\NormalTok{, }
         \AttributeTok{intitule=}\StringTok{\textasciigrave{}}\AttributeTok{Intitulé de l\textquotesingle{}acte}\StringTok{\textasciigrave{}}\NormalTok{, }
         \AttributeTok{date=}\StringTok{\textasciigrave{}}\AttributeTok{Date d\textquotesingle{}affichage}\StringTok{\textasciigrave{}}\NormalTok{)}
\end{Highlighting}
\end{Shaded}

\subsection{Extraction d'informations}\label{extraction-dinformations}

\subsubsection{L'année de publication des
actes}\label{lannuxe9e-de-publication-des-actes}

La première extraction que l'on va faire concerne la date de publication
des arrêtés afin de voir la période couverte par notre base de données.
La colonne \texttt{arretes\$date} contient l'information qui nous
intéresse avec une granularité temporelle cependant trop fine.

\begin{Shaded}
\begin{Highlighting}[]
\NormalTok{arretes}\SpecialCharTok{$}\NormalTok{date[}\DecValTok{1}\SpecialCharTok{:}\DecValTok{10}\NormalTok{]}
\end{Highlighting}
\end{Shaded}

\begin{verbatim}
 [1] "2021-12-17" "2021-12-17" "2021-12-17" "2021-12-20" "2022-02-15"
 [6] "2022-02-16" "2022-02-14" "2022-02-21" "2022-03-07" "2021-11-16"
\end{verbatim}

Pour extraire l'information qui nous intéresse, on peut se focaliser sur
les 4 premiers caractères de la date qui correspondent à l'année.

\begin{Shaded}
\begin{Highlighting}[]
\NormalTok{arretes}\SpecialCharTok{$}\NormalTok{annee }\OtherTok{\textless{}{-}} \FunctionTok{substr}\NormalTok{(arretes}\SpecialCharTok{$}\NormalTok{date,}\DecValTok{1}\NormalTok{, }\DecValTok{4}\NormalTok{)}

\NormalTok{arretes\_annee }\OtherTok{\textless{}{-}}\NormalTok{ arretes }\SpecialCharTok{\%\textgreater{}\%} 
  \FunctionTok{group\_by}\NormalTok{(annee) }\SpecialCharTok{\%\textgreater{}\%} 
  \FunctionTok{summarise}\NormalTok{(}\AttributeTok{n=}\FunctionTok{n}\NormalTok{())}
\end{Highlighting}
\end{Shaded}

On pourra représenter cette information par un graphique.

\begin{Shaded}
\begin{Highlighting}[]
\FunctionTok{library}\NormalTok{(ggplot2)}

\FunctionTok{ggplot}\NormalTok{()}\SpecialCharTok{+}
  \FunctionTok{geom\_col}\NormalTok{(}\AttributeTok{data =}\NormalTok{ arretes\_annee, }\FunctionTok{aes}\NormalTok{(}\AttributeTok{x=}\NormalTok{annee, }\AttributeTok{y=}\NormalTok{n))}\SpecialCharTok{+}
  \FunctionTok{labs}\NormalTok{(}\AttributeTok{y=}\StringTok{"Nombre d\textquotesingle{}actes"}\NormalTok{, }
       \AttributeTok{x=}\StringTok{"Année"}\NormalTok{)}
\end{Highlighting}
\end{Shaded}

\pandocbounded{\includegraphics[keepaspectratio]{01_prise_en_main_files/figure-pdf/unnamed-chunk-8-1.pdf}}

\subsubsection{Les types d'acte}\label{les-types-dacte}

La difficulté principale de l'exercice réside dans l'extraction
d'informations non-structurées des chaînes de caractères qui composent
le titre des actes. Pour commencer ce travail, on va d'abord en
visualiser quelques-unes afin de voir si on peut dégager des éléments de
structuration.

\begin{Shaded}
\begin{Highlighting}[]
\NormalTok{arretes}\SpecialCharTok{$}\NormalTok{intitule[}\FunctionTok{sample}\NormalTok{(}\DecValTok{1}\SpecialCharTok{:}\FunctionTok{nrow}\NormalTok{(arretes), }\DecValTok{5}\NormalTok{)]}
\end{Highlighting}
\end{Shaded}

\begin{verbatim}
[1] "Arrêté temporaire de circulation pour travaux-Quartier n°4 - Saint Augustin/Tauzin/A. Dupeux-du 17 au 23 Rue Léon Roches des deux côtés de la voie-1 - Interdiction de stationnement"
[2] "Arrêté temporaire de circulation pour travaux-Quartier n°3 - Bordeaux Centre-Rue Condillac, du 37 jusqu'à la Rue Buffon-1 - Interdiction de stationnement"                           
[3] "DECISION DE TARIFS - FIXATION DE TARIFS POUR LA VENTE DE PRODUITS EN BOUTIQUE AU CAPC"                                                                                               
[4] "Arrêté de mise en service de grue-Quartier n°6 - Bordeaux Sud- Rue Amédée Saint-Germain, ilot 9.12 \"Passage Saint Germain\"-Mise en service de grue (Bordeaux)"                     
[5] "ARRÊTÉ TEMPORAIRE DE CIRCULATION - Quartier n°6 - Bordeaux Sud-Face au 82 et jusqu'au 58 Rue des Terres de Borde côté SNCF-1 - Neutralisation de voie"                               
\end{verbatim}

On voit que le type d'acte figure en début de chaîne, séparé du reste
par un tiret (\texttt{-}). On peut mobiliser une \textbf{expression
régulière} (\textbf{\emph{regex}}) pour extraire du texte qui suit un
pattern identifié.

\begin{tcolorbox}[enhanced jigsaw, toptitle=1mm, colbacktitle=quarto-callout-note-color!10!white, opacitybacktitle=0.6, bottomrule=.15mm, breakable, arc=.35mm, left=2mm, opacityback=0, rightrule=.15mm, colframe=quarto-callout-note-color-frame, coltitle=black, colback=white, leftrule=.75mm, bottomtitle=1mm, title=\textcolor{quarto-callout-note-color}{\faInfo}\hspace{0.5em}{🔍 Les expressions régulières (regex)}, toprule=.15mm, titlerule=0mm]

Une \textbf{regex} (\emph{regular expression}) est une \textbf{chaîne de
caractères spéciale} qui permet de \textbf{décrire un motif} (ou
\emph{pattern}) dans du texte.\\
Elles sont très utiles pour :

\begin{itemize}
\item
  rechercher (\texttt{str\_detect()}),
\item
  extraire (\texttt{str\_extract()}),
\item
  remplacer (\texttt{str\_replace()}),
\item
  ou découper (\texttt{str\_split()})
\end{itemize}

des morceaux de texte selon une \textbf{règle de forme}, et non un mot
exact.

En R, on les utilise souvent avec le package \textbf{\{stringr\}}.

\end{tcolorbox}

\begin{tcolorbox}[enhanced jigsaw, toptitle=1mm, colbacktitle=quarto-callout-tip-color!10!white, opacitybacktitle=0.6, bottomrule=.15mm, breakable, arc=.35mm, left=2mm, opacityback=0, rightrule=.15mm, colframe=quarto-callout-tip-color-frame, coltitle=black, colback=white, leftrule=.75mm, bottomtitle=1mm, title=\textcolor{quarto-callout-tip-color}{\faLightbulb}\hspace{0.5em}{⚙️ Exemple : extraire ce qu'il y a avant un tiret}, toprule=.15mm, titlerule=0mm]

Supposons que nous ayons un vecteur de chaînes de caractères :

\begin{Shaded}
\begin{Highlighting}[]
\FunctionTok{library}\NormalTok{(stringr)}

\NormalTok{x }\OtherTok{\textless{}{-}} \FunctionTok{c}\NormalTok{(}\StringTok{"Bordeaux {-} Gironde"}\NormalTok{, }\StringTok{"Lyon {-} Rhône"}\NormalTok{, }\StringTok{"Marseille {-} Bouches{-}du{-}Rhône"}\NormalTok{)}
\end{Highlighting}
\end{Shaded}

\textbf{Étape 1 : la regex}

Le motif à décrire est : ``tous les caractères avant un tiret''

👉 En regex, cela s'écrit : \texttt{\^{}{[}\^{}-{]}+}

\begin{itemize}
\item
  \texttt{\^{}} → début de la chaîne
\item
  \texttt{{[}{]}} → ensemble de caractères
\item
  \texttt{{[}\^{}\ {]}} → tout sauf ce qui est entre les crochets
\item
  \texttt{+} → répète 1 ou plusieurs fois
\end{itemize}

\textbf{Étape 2 : l'appliquer avec} \texttt{str\_extract()}

\begin{Shaded}
\begin{Highlighting}[]
\FunctionTok{str\_extract}\NormalTok{(x, }\StringTok{"\^{}[\^{}{-}]+"}\NormalTok{)}
\end{Highlighting}
\end{Shaded}

\begin{verbatim}
[1] "Bordeaux "  "Lyon "      "Marseille "
\end{verbatim}

On peut ensuite \textbf{supprimer les espaces} avec \texttt{str\_trim()}
:

\begin{Shaded}
\begin{Highlighting}[]
\FunctionTok{str\_trim}\NormalTok{(}\FunctionTok{str\_extract}\NormalTok{(x, }\StringTok{"\^{}[\^{}{-}]+"}\NormalTok{))}
\end{Highlighting}
\end{Shaded}

\begin{verbatim}
[1] "Bordeaux"  "Lyon"      "Marseille"
\end{verbatim}

\end{tcolorbox}

Appliqué aux titres des actes publiés par la ville de Bordeaux, on aura
donc :

\begin{Shaded}
\begin{Highlighting}[]
\NormalTok{arretes}\SpecialCharTok{$}\NormalTok{type }\OtherTok{\textless{}{-}} \FunctionTok{str\_trim}\NormalTok{(}\FunctionTok{tolower}\NormalTok{(}\FunctionTok{str\_extract}\NormalTok{(arretes}\SpecialCharTok{$}\NormalTok{intitule, }\StringTok{"\^{}[\^{}{-}]+"}\NormalTok{)))}

\FunctionTok{sample}\NormalTok{(arretes}\SpecialCharTok{$}\NormalTok{type, }\DecValTok{10}\NormalTok{)}
\end{Highlighting}
\end{Shaded}

\begin{verbatim}
 [1] "arrêté temporaire de circulation"             
 [2] "arrêté temporaire de circulation"             
 [3] "arrêté temporaire de circulation"             
 [4] "mise en securite urgente"                     
 [5] "arrêté de délégation de signature aux élus"   
 [6] "arrêté temporaire de circulation pour travaux"
 [7] "arrêté temporaire de circulation"             
 [8] "arrêté de mise en place de grue"              
 [9] "arrêté temporaire de circulation"             
[10] "arrêté temporaire de circulation"             
\end{verbatim}

On peut également compléter cette identification par une variable qui
repère quand la chaîne de caractère correspondant à l'intitulé de l'acte
comporte ``\emph{immeuble}'' ou ``\emph{habitation}''.

\begin{Shaded}
\begin{Highlighting}[]
\NormalTok{arretes }\OtherTok{\textless{}{-}}\NormalTok{ arretes }\SpecialCharTok{\%\textgreater{}\%} 
  \FunctionTok{mutate}\NormalTok{(}\AttributeTok{immeuble=}\FunctionTok{grepl}\NormalTok{(}\StringTok{"immeuble|habitation"}\NormalTok{, }\FunctionTok{tolower}\NormalTok{(intitule)))}
\end{Highlighting}
\end{Shaded}

Si on combine les deux colonnes, on peut voir quels types d'arrêtés
mentionnent fréquemment des immeubles ou des habitations.

\begin{Shaded}
\begin{Highlighting}[]
\NormalTok{arretes\_type\_immeubles }\OtherTok{\textless{}{-}}\NormalTok{ arretes }\SpecialCharTok{\%\textgreater{}\%} 
  \FunctionTok{group\_by}\NormalTok{(type, immeuble) }\SpecialCharTok{\%\textgreater{}\%} 
  \FunctionTok{summarise}\NormalTok{(}\AttributeTok{n=}\FunctionTok{n}\NormalTok{()) }\SpecialCharTok{\%\textgreater{}\%} 
  \FunctionTok{arrange}\NormalTok{(}\FunctionTok{desc}\NormalTok{(immeuble), }\FunctionTok{desc}\NormalTok{(n))}

\NormalTok{arretes\_type\_immeubles }\SpecialCharTok{\%\textgreater{}\%} \FunctionTok{head}\NormalTok{(}\AttributeTok{n =} \DecValTok{10}\NormalTok{) }\SpecialCharTok{\%\textgreater{}\%}\NormalTok{ knitr}\SpecialCharTok{::}\FunctionTok{kable}\NormalTok{()}
\end{Highlighting}
\end{Shaded}

\begin{longtable}[]{@{}llr@{}}
\toprule\noalign{}
type & immeuble & n \\
\midrule\noalign{}
\endhead
\bottomrule\noalign{}
\endlastfoot
mainlevee & TRUE & 277 \\
urgence securite immeubles l.2212 & TRUE & 172 \\
arrêté permanent de circulation et de stationnement & TRUE & 165 \\
arrete de mise en securite urgente & TRUE & 54 \\
mise en securite urgente & TRUE & 54 \\
arrete de mise en securite ordinaire & TRUE & 52 \\
mise en securite ordinaire & TRUE & 51 \\
arrêté de mise en securite ordinaire & TRUE & 48 \\
arrêté temporaire de circulation & TRUE & 44 \\
arrêté mise en securite urgente & TRUE & 37 \\
\end{longtable}

Après une inspection manuelle, on voit que les types d'arrêtés relatifs
à la dégradation des immeubles sont les suivants :

\begin{Shaded}
\begin{Highlighting}[]
\NormalTok{arretes\_selec }\OtherTok{\textless{}{-}}\NormalTok{ arretes\_type\_immeubles}\SpecialCharTok{$}\NormalTok{type[}\FunctionTok{c}\NormalTok{(}\DecValTok{1}\NormalTok{,}\DecValTok{2}\NormalTok{,}\DecValTok{4}\NormalTok{,}\DecValTok{5}\SpecialCharTok{:}\DecValTok{8}\NormalTok{,}\DecValTok{10}\NormalTok{,}\DecValTok{11}\NormalTok{,}\DecValTok{13}\SpecialCharTok{:}\DecValTok{15}\NormalTok{,}\DecValTok{22}\NormalTok{,}\DecValTok{54}\NormalTok{,}\DecValTok{99}\NormalTok{,}\DecValTok{100}\NormalTok{,}\DecValTok{106}\NormalTok{)]}
\end{Highlighting}
\end{Shaded}

On crée donc un nouvel objet avec uniquement les arrêtés relatifs à des
immeubles.

\begin{Shaded}
\begin{Highlighting}[]
\NormalTok{perils }\OtherTok{\textless{}{-}}\NormalTok{ arretes }\SpecialCharTok{\%\textgreater{}\%} 
  \FunctionTok{filter}\NormalTok{(type }\SpecialCharTok{\%in\%}\NormalTok{ arretes\_selec)}
\end{Highlighting}
\end{Shaded}

Enfin, on ne veut travailler que sur la distribution spatiale des
périls, peu importe s'ils ont été traités ou non. On retire donc des
données tout ce qui relève des mainlevées.

\begin{Shaded}
\begin{Highlighting}[]
\NormalTok{perils }\OtherTok{\textless{}{-}}\NormalTok{ perils }\SpecialCharTok{\%\textgreater{}\%} 
  \FunctionTok{filter}\NormalTok{(}\SpecialCharTok{!}\FunctionTok{grepl}\NormalTok{(}\StringTok{"mainlev"}\NormalTok{, }\FunctionTok{tolower}\NormalTok{(intitule)))}
\end{Highlighting}
\end{Shaded}

\subsubsection{La localisation des
arrêtés}\label{la-localisation-des-arruxeatuxe9s}

Pour localiser les arrêtes, il va falloir extraire les adresses et les
géocoder. On commence pour cela par définir la liste des types de voies
que l'on souhaite détecter :

\begin{Shaded}
\begin{Highlighting}[]
\NormalTok{types\_voie }\OtherTok{\textless{}{-}} \FunctionTok{c}\NormalTok{(}\StringTok{"rue"}\NormalTok{, }\StringTok{"avenue"}\NormalTok{, }\StringTok{"boulevard"}\NormalTok{, }\StringTok{"bd"}\NormalTok{, }\StringTok{"chemin"}\NormalTok{, }\StringTok{"impasse"}\NormalTok{, }\StringTok{"place"}\NormalTok{, }\StringTok{"quai"}\NormalTok{, }\StringTok{"allée"}\NormalTok{, }\StringTok{"cours"}\NormalTok{)}
\end{Highlighting}
\end{Shaded}

On construit ensuite une expression régulière qui cherche une structure
du type\\
numéro + (optionnellement ``bis'', ``ter'', etc.) + type de voie + nom
de voie :

\begin{Shaded}
\begin{Highlighting}[]
\NormalTok{pattern }\OtherTok{\textless{}{-}} \FunctionTok{paste0}\NormalTok{(}
  \StringTok{"}\SpecialCharTok{\textbackslash{}\textbackslash{}}\StringTok{b}\SpecialCharTok{\textbackslash{}\textbackslash{}}\StringTok{d+}\SpecialCharTok{\textbackslash{}\textbackslash{}}\StringTok{s?(?:bis|ter|quater)?}\SpecialCharTok{\textbackslash{}\textbackslash{}}\StringTok{s+("}\NormalTok{, }
  \FunctionTok{paste}\NormalTok{(types\_voie, }\AttributeTok{collapse =} \StringTok{"|"}\NormalTok{), }
  \StringTok{")}\SpecialCharTok{\textbackslash{}\textbackslash{}}\StringTok{s+[[:alpha:]][[:alnum:]}\SpecialCharTok{\textbackslash{}\textbackslash{}}\StringTok{s}\SpecialCharTok{\textbackslash{}\textbackslash{}}\StringTok{{-}\textquotesingle{}]+"}
\NormalTok{)}
\end{Highlighting}
\end{Shaded}

On applique cette regex à la colonne \texttt{intitule} pour extraire les
adresses :

\begin{Shaded}
\begin{Highlighting}[]
\NormalTok{adresses }\OtherTok{\textless{}{-}} \FunctionTok{str\_extract}\NormalTok{(}\FunctionTok{tolower}\NormalTok{(perils}\SpecialCharTok{$}\NormalTok{intitule), }\FunctionTok{regex}\NormalTok{(pattern, }\AttributeTok{ignore\_case =} \ConstantTok{TRUE}\NormalTok{) )}

\FunctionTok{head}\NormalTok{(adresses)}
\end{Highlighting}
\end{Shaded}

\begin{verbatim}
[1] "9 rue leyteire"                      NA                                   
[3] "3 quai deschamps à bordeaux "        "86 rue sainte catherine à bordeaux "
[5] "329 avenue thiers à bordeaux "       "329 avenue thiers "                 
\end{verbatim}

Enfin, on nettoie le résultat en retirant la mention ``à Bordeaux'' et
les espaces inutiles :

\begin{Shaded}
\begin{Highlighting}[]
\NormalTok{perils}\SpecialCharTok{$}\NormalTok{adresse }\OtherTok{\textless{}{-}} \FunctionTok{trimws}\NormalTok{(}\FunctionTok{gsub}\NormalTok{(}\StringTok{"à bordeaux"}\NormalTok{, }\StringTok{""}\NormalTok{, adresses))}
\end{Highlighting}
\end{Shaded}

Le résultat est une colonne \texttt{adresse} propre, contenant les
adresses détectées automatiquement dans le texte.

\subsection{La transformation de données tabulaires en données
spatiales}\label{la-transformation-de-donnuxe9es-tabulaires-en-donnuxe9es-spatiales}

\subsubsection{Géocodage}\label{guxe9ocodage}

\subsubsection{Transformation en objet
spatial}\label{transformation-en-objet-spatial}




\end{document}
